\documentclass[crop,border=0.1cm]{standalone} % Create smallest PDF canvas

\pagestyle{empty}                         % No header or footer

\usepackage{amsmath}                      % Can use align environment
\usepackage[usenames,dvipsnames]{xcolor}  % Extra colours

\usepackage{pgfplots}                     % PGF/TiKZ plotting package
\pgfplotsset{compat=newest}               % Should use this

\newcommand{\unit}[1]{\,#1 }              % Shorthand for units
\newcommand{\mM}{\unit{mM}}
\newcommand{\mW}{\unit{mW}}

\usepackage[sc]{mathpazo}

\begin{document}
\begin{tikzpicture}

  \begin{axis}[
    footnotesize,               % Footnote sized text
    width  = 4.65015in,
    height = 200pt,
    legend style = {
      draw = none,              % Do not draw a box around the legend
    },
    xmin = -2,                 % Use certain x values. Required even with xtick
    xmax = 60,                 % Use certain x values. Required even with xtick
    xtick = {0,5,...,60},  % Define how often the ticks should happen
    minor x tick num = 4,       % How many minor ticks between major ticks
    ymin = 96,
    ymax = 100,
    ytick = {96,97,...,100},
    minor y tick num = 4,
    xlabel = Time (Minutes),
    ylabel = Normalised Intensity (\% of max)
  ]

    \foreach \x in {1,2,...,12}{\addplot[RoyalBlue] file{\x.csv};}

  \end{axis}
\end{tikzpicture}
\end{document}

