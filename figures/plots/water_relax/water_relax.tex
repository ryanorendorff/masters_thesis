\documentclass[crop,border=0.1cm]{standalone} % Create smallest PDF canvas

\pagestyle{empty}                         % No header or footer

\usepackage{amsmath}                      % Can use align environment
\usepackage[usenames,dvipsnames]{xcolor}  % Extra colours

\usepackage{pgfplots}                     % PGF/TiKZ plotting package
\pgfplotsset{compat=newest}               % Should use this

\usepackage[osf,sc]{mathpazo}
%\linespread{1.05} % a bit more for Palatino


\begin{document}
\begin{tikzpicture}

\pgfplotstableread[col sep=comma]{relax.csv}\wrelax

  \begin{axis}[
    footnotesize,               % Footnote sized text
    width  = 4.65015in,
    height = 200pt,
    legend style = {
      at={(0.99,0.5)},
      anchor=east,
      %legend pos=south east
      draw = none              % Do not draw a box around the legend
    },
    xmin = 0,                 % Use certain x values. Required even with xtick
    xmax = 95,                 % Use certain x values. Required even with xtick
    xtick = {0,5,...,95},  % Define how often the ticks should happen
    minor x tick num = 4,       % How many minor ticks between major ticks
    xtick pos = left,
    ymin = 0.50,
    ymax = 1.02,
    ytick = {0.50,0.60,0.7,0.8,0.9,1.0},
    minor y tick num = 4,
    xlabel = Time (seconds),
    ylabel = Normalised Intensity
  ]

  \addplot[RoyalBlue] table{\wrelax};
  \addlegendentry{Intensity Data}
  \addplot[Maroon,domain=35:95,dashed] {(1-exp(-(x-35)/13))*(.36578)+0.63422};
  \addlegendentry{Fit using $1-e^{-t/\tau}$}
  \draw[Orange] (axis cs:0,0.51) -- (axis cs:36,0.51) node[midway,anchor=south]{\emph{\footnotesize{Injection Phase}}};
  \draw[RoyalPurple] (axis cs:36,0.51) -- (axis cs:95,0.51) node[midway,anchor=south]{\emph{\footnotesize{Relaxation Phase}}};

  \end{axis}
\end{tikzpicture}
\end{document}

