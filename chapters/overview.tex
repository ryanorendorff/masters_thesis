\chapter{Overview of Spectroscopy}\label{ch:introduction}

Spectroscopy is a field in science that uses our understanding of the way
that matter interacts with energy, including the emission and absorption
of energy from matter. The fact that matter absorbs, emits, refracts, and
deflects energy allows us to probe a substance with minimal interference,
while still allowing a great variety of information to be derived, such as
the chirality of a molecule, the energy levels of an atom's
electronic structure, and the concentration of various chemicals in any type of solution.



In this chapter we will discuss a variety of absorption spectroscopy
techniques, including single pass absorption spectroscopy, broadband
absorption spectroscopy, multipass \ac{TDL} white cell setups, \ac{PAS},
and \ac{CEAS}. These can be grouped approximately into basic
absorption, broadband absorption and multipass absorption techniques, the last
two of which can be combined to make ultra sensitive spectroscopy techniques.


\section{Theory behind Absorption Spectroscopy}\label{sec:theory}

Absorption spectroscopy is founded on a simple principle: if matter absorbs
light, then we can detect this energy transfer via the law of the conservation
of energy. One of the simplest ways to measure this absorption event is to
count the number of photons that went into a piece of matter and how many
came out. In this way we can determine how much energy was \emph{absorbed}
by the matter. This absorption is correlated to the concentration of the
absorbing material, and is unique to the material and the material's state.

In this report we will be primary interested in determining concentration
of absorbers in a substance, and the electronic energy structure of the
absorbers.

\subsection{Beer-Lambert Law}\label{subsec:beer}

Mathematically, the problem of determining the concentration of an absorber
based on the light lost in a material is relatively simple. By measuring the
amount of light of a particular wavelength $\lambda$ entering a material, with
intensity $I$, and the amount of light exiting $I_0$, we can easily relate
the intensities to each other to determine how much light was lost due to
absorption.

For most physical cases, absorption follows a logarithmic dependence
with the concentration of the absorber. Therefore, using the simple
equation\marginpar{$A =$  absorption (percent), \\$I_0 = $
intensity before the cavity ($\text{W}/\text{m}^2$),\\$I =$
intensity after the cavity ($\text{W}/\text{m}^2$).}

\begin{align*}
  A(\lambda)=-\log\left(\frac{I}{I_0}\right)_\lambda
\end{align*}

This can then be related to the absorption of a particular chemical substance
by using some properties of that substance and the distance the light
travelled through the substance.\marginpar{$\epsilon = $ molar
absorptivity (M$^{-1}$cm$^{-1}$),\\ $\alpha =$ absorption coefficient
(cm$^{-1}$), $c = $ concentration of the absorber (M),\\$l =$ path
length (cm).}

\begin{align}
  A(\lambda) = \epsilon(\lambda) c l = \alpha(\lambda) l = -\log\left(\frac{I}{I_0}\right)_\lambda
\end{align}

This is known as Beer Lambert's Law, which relates the loss of light in a
material to the concentration of absorbers inside that material.  This law is
stated in a variety of different ways, including using the natural logarithm
instead of a base ten logarithm. \marginpar{Index of refraction $\mathscr{N} =
n + i\kappa$}Critically, the introduced variable $\alpha$ is defined physically
as a function of the imaginary portion of the complex index of refraction.

Beer-Lambert's Law can also be defined for multiple absorbers by simply adding the absorption coefficients of all of the absorbers together.
\begin{align*}
  A(\lambda) = l\sum\alpha(\lambda)_i
\end{align*}

While Beer-Lambert's Law is convenient, it is frustratingly bound by a number of factors.
\begin{itemize}
  \item Little tolerance to other light losing effects such as scattering.
  \item Optical saturation limits.
  \item For most desirable analyses, a homogeneous mixture of the absorbers.
  \item Collimated light sources to allow the path length to be constant.
\end{itemize}

Even with these limitations, the technique of measuring intensity loss to
determine absorber concentration is used in many analytical devices. One of the
most familiar is a pulse oximeter, which measures blood oxygenation levels
using two laser diodes placed on the index finger of a patient, with
photodiodes on the other side of the finger to measure the intensity
loss.\marginpar{Pulse oximeters can be used together with other technologies to
acquire physiological parameters across the entire body: see the Esoma
project. \url{http://www.rdodesigns.com}}

\subsection{Electronic Transitions}
Knowing how to measure concentration using absorption measurements is useful,
but with monochromatic light it is one of the few parameters that can be
measured about an absorber. However, if we can derive a method to collect
information about the absorption of many different wavelengths we can start to
derive information about the energy states of the electrons inside the absorber.\marginpar{In this section we will be talking about atoms but the concepts apply to molecules as well}

%% OMG THIS PARAGRAPH SUCKS
An introductory level quantum chemistry course introduces the concept that
electrons in an atom, when bound to a set of nuclei, are forced into set
quantum states, each of which has a certain energy level\marginpar{some quantum
  states are degenerate, so each quantum state does not necessarily have a
unique energy}. An atom is usually found in its unexcited, stable state known
as the ground state. In this state, the atom has the potential to jump to a
higher energy (and usually less stable) configuration when external energy is
applied to the atom.  This input energy can come in the form of electromagnetic
radiation. This radiation must have the same energy as one of the quantum state
transitions of the atom for the radiation to be absorbed. During the absorption
process, the electron that absorbed the energy enters an excited state. \marginpar{There are a few other nonlinear absorption/emission processes that can happen as well.} This electron then eventually releases the energy it absorbed by either emitting a photon or by thermally releasing energy in the form of a collision.

In the process of absorbing a photon to reach an excited state, the atom has
revealed bits of information about itself. Since the energy transitions are
quantised, we can theoretically guess where absorption in the electromagnetic
spectrum should occur. \marginpar{Hartree-Fock approximations and Density
Functional Theory (DFT)} In fact, quantum chemistry is a field dedicated
to determining the quantum states of molecular systems over time, using
approximations to the Schr\"odinger equations.

\subsection{Line Broadening}

Under perfect theoretical conditions, with perfectly still particles,
absorption lines would be fiendishly thin due to the fact that only certain
quantised energy packets would create an excited condition. In reality,
absorption lines are quite broad, on the order of a few nanometers in liquid.
This broadening occurs due primarily to two factors: the Doppler effect and
collision based broadening (also known as Lorentzian broadening).

Doppler based broadening is just the Doppler effect applied to moving
particles. If a particle at rest will see a incoming photon of light having a
certain frequency. If this particle is instead moving away from the photon,
the frequency of the light will appear to the particle as a lower frequency,
as the distance between the crests and troughs of the wave is greater in the
particles inertial frame. Conversely, if the particle is moving towards the
photon, the photon will appear to have a higher frequency. The particle in
question will still absorb at the same quantised energy, but now photons with
greater and lesser energy than the transition will appear to have the correct
energy, broadening the range of wavelengths that the particle absorbs at.

Collision based broadening leads to a similar effect of a broadening of
acceptable photon energies, but through a thermal pathway. As particles in
the solution collide into each other, they acquire some energy that puts them
in a state higher than their ground energetic state. As such, less energy is
required by the photon to cause an absorption event, leading to the absorption
of lower energy photons.

Both of these effects act on electronic transitions in absorption measurements
performed in aqueous environments, leading to very broad absorption peaks.
This can lead to difficulty in analysis in two different ways.

First, if two absorption peaks are too close together, they will appear as
one combined absorption peak. This causes a loss of information, as it is
impossible in most cases to calculate the absorption due to one transition
versus its nearby neighbor. This type of event does not normally occur in
a single molecule, as the accepted electronic transitions are spaced out
energetically due to the higher energy required to jump to a higher molecular
orbital. This collision of signals does become a problem when multiple
absorbers are present in a solution, often leading to signals that cannot be
decoupled from each other.

The second difficulty broadening causes in analysis is the understanding of
the hyperfine structure of certain transitions. When an electron is excited to
a higher energy state, it can often be excited into a multiplet of rotational
and vibrational energy states. On top of this, many degenerate energy states
can exist for one transition, which will only be seen if an external electric
or magnetic field causes the fields to split (via the Stark effect). These
different states are separated from each other by minute differences in
energy. These slightly different transitions are normally difficult to detect,
and are, for all intents and purposes, unobservable in liquid solutions.

\subsection{Scattering Losses}

In addition to broadening schemes, liquid absorption measurements fall victim
to the fact that Beer-Lambert type experiments cannot differentiate between
light lost due to scattering versus due to absorption.  In gas phase
measurements, scattering is often not a concern due to the density of the gas.
However, with the higher probability of a photon being scattered in liquids,
the assumption that scattering plays a minor role in light loss is no longer
valid. This leads to higher error and required higher initial intensities into
an optical cavity, as much of the light will be lost just by trying to
traversing the cavity.

The loss of light due to scattering also restricts the types of solutions that
can be observed. If a solution is turbid to begin with, it may be difficult to
find a powerful enough light source that can push enough photons through the
sample to detect a meaningful difference due to an absorption event. Similarly,
a solution that becomes turbid during a measurement through either chemical
reactions or colloidally will be an impractical candidate for absorption
measurements, as the loss of light due to scattering changes over the time of
the measurements. This problem especially becomes a problem in experiments
designed to determine reaction kinematics, as the absorption measurements over
time will not be comparable due to the shifting noise caused by scattering.

Light can also be lost due to changes in the index of refraction for the
solution. Many gas solutions are dilute enough that the index of refraction of
the medium does not change significantly between measurements and blanks. This
scenario is different in liquids, where the index of refraction can change
significantly depending on the solvent and even the concentration of the
solute. This change in the index of refraction does not alter the absorption
events themselves, but presents a difficulty when attempting to align an
optical system. In the simple case of a single pass absorption, this effect can
lead to a loss of signal due to divergent beams that is dependent on the
wavelength, leading to a nonlinear error in the observed absorption spectrum.
In multipass systems, the misalignment of an optical cavity due to the change
in the index of refraction can preferentially support and discourage certain
modes, leading to a similar nonlinear error in an absorption spectrum. These
divergent beam problems can be mitigated by using a collimated source that
enters perpendicular to the cavity walls, negating the effects of the changes
in refraction. However, in practice it is quite difficult to create a perfect
perpendicular alignment.

Even though broadening, scattering and changes in the index of refraction
can cause high and nonlinear errors to occur in absorption measurements,
these effects can be mitigated by a careful choice of the layout of the
optical system and the choice of the analyte. Experiment that measure clear
solutions with minimal changes in index of refraction from a blank measurement
are candidates good candidates for absorption spectroscopy, given that the
experimenter does not require an understanding of the hyperfine structure of
the absorber in question.

\section{Broadband Absorption Spectroscopy}


\subsection{Tunable Diode Lasers}
NCAR FTW!

\subsection{Supercontinuum Lasers}
They hurt your eyes!

\subsection{LEDs}
They are very poorly collimated and diffuse!

\section{Multipass Techniques}
I like to visit the same place multiple times.

\subsection{White and Herriot Cells}
Sometimes when I pass by I do different things.

\subsection{Cavities}
Sometimes I just want to do the same thing over and over.

\section{Ring down}
What's that sonny?

\section{Summary}
Review chapter here.
