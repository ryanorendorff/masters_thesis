\chapter{Calibrations}\label{chap:calibration}

\acl{BBCEAS} has a simpler setup and lower cost in comparison to \ac{CRDS}, but
does require a few calibration procedures in order to acquire accurate
absorption spectra. The basic calibrations that should be done on a
\ac{BBCEAS} instrument are a measure of the absorption limits where the signal
is linear, an analysis of the error of the detected intensity signal, and the
reflectivity of the mirrors inside the cavity.

To find the absorption coefficient values under which an \ac{BBCEAS}
experiment results in a linear correlation with concentration, it is common to
test the instrument using a strong absorber. One such absorber is Rhodamine
6G, a dye that can be used to create a dye laser or as a fluorescent marker.

The error in the detection of the intensity comes down to two important
sources: the error in the \ac{CCD} and the intensity fluctuation over time of
the light source. The intensity fluctuation is especially important, as it is
possible (and in some cases, extremely likely) for the intensity of the light
into the cavity to change between the measurement of the blank and the
measurement of the sample. This fluctuation at best provides a false
understanding of the detection limit of the setup and, in the case of high
variability in the input source, silently invalidates all data acquired with
the technique.

Finally, the mirror reflectivity is required for any of the equations of the
absorption coefficient or its standard deviation. Giving a guess on the mirror
reflectivity at a particular wavelength based on the average mirror
reflectivity can lead to inaccurate absorption coefficient calculations and
alter the shape of the calculated spectrum. This case is more common than one
may expect, as the multiple layers of thin films deposited on the mirror's
surface introduces characteristic ripples in the mirror reflectivity as a
function of wavelength.

This chapter will discuss these calibrations to characterise the \ac{BBCEAS}
instrument used in this report.

\section{Limits of linearity using Rhodamine 6G}\label{sec:rhodamine}

There are two important considerations for regions where the concentration of
rhodamine correlates linearly with the concentration of an absorber. One
consideration is the calculation of absorption using $A = \alpha  l$, where
$l$ is the length of the cavity times the number of passes of the light
between the mirrors. It is possible to measure an $\alpha$ value that leads to
an absorption of greater than one, which of course is illogical. This
phenomenon occurs when the approximation for the path length is no longer
defined by $\tfrac{1-R}{d}$. In the particular setup used, where the mirror
reflectivity $R=0.99$, the approximate upper bound of the linear region cannot
exceed $\alpha = 1.84\cdot10^{-3}\,\text{cm}^{-1}$.

\begin{figure}
\begin{center}
\includegraphics[width=\textwidth]{figures/Rh6G_absorption_cross_section_and_linearity}
\end{center}
\caption{Rhodamine 6G Spectrum}
\label{fig:rh6g}
\end{figure}

While the upper bound to the linearity of the \ac{BBCEAS} equation is useful
for determining when experimental results may be strange, a tighter bounding
set is useful to determine what sorts of concentrations would be best measured
by the \ac{BBCEAS} technique. Unfortunately, the only way to measure where
absorption coefficients correlate linearly with concentration is to acquire
spectra from many different concentrations and plot the absorption at the peak
wavelength as a function of concentration.\marginpar{Get linearity figure}
This is shown below.

As can be seen in the figure, the lower the concentration, the better we can
fit a linear regression to the data acquired. As such, a lower bound is better
defined as the limit of detection, instead of the linearity at low
concentrations.

One can see that there is only about an order of magnitude where the
absorption coefficient can be effectively measured. This aspect is largely
ignored in the literature except by some astute groups.

One problem with this technique is that the concentration of an absorber that
corresponds with a particular value of the absorption coefficient is defined
by the absorber itself. Strong absorbers will have smaller concentrations
where the calculated absorption coefficient is linear. Weak absorbers have an
advantage is greater detection ranges under this regime, but suffer from a
lower sensitivity to the concentration as small changes in intensity correlate
to larger changes in concentration, in comparison to strong absorbers.

As a guide to the detectable concentration ranges, it is possible to calculate
the molar emissivity $\epsilon$ range of a \ac{BBCEAS} setup by dividing
the absorption coefficient value at the upper bound by the concentration
it represents. \marginpar{One could potentially use, for simple molecules,
quantum chemistry and Hartree-Fock approximations to estimate at what energies
light would be absorbed and approximately how strong such a transition would
be, but this technique is far from simple and likely takes longer than just
trying to measure several concentrations of the absorber using \ac{BBCEAS}
iteratively.} Then, if one knows the molar emissivity of an absorber it
is possible to determine a rough upper bound of the concentration limit of
detection. Unfortunately many substances of interest have unknown or poorly
characterised absorption spectra in terms of molar emissivity in the liquid
phase, and there is no convenient methods of determining valid concentration
ranges.

\section{Intensity Fluctuations in BBCEAS measurements}\label{sec:laser_fluc}

\subsection{Intensity fluctuations due to light source stability}

\begin{figure}[h!]
\begin{center}
\includegraphics[width=\textwidth]{figures/change_in_intensity_of_576_5nm_over_one_hour.png}
\end{center}
\caption{Laser Fluctuation over time}
\label{fig:laser_fluc}
\end{figure}

\ac{BBCEAS} measurements, while based on \ac{CRDS}, do not share the
benefit of being immune to intensity fluctuations in the source. This is
because \ac{CRDS} takes the measurement of the intensity of an individual
pulse of light, whereas \ac{BBCEAS} measures the average steady state ring down
intensity, which represents an average of the light intensity ring down times
across multiple pulses (for pulsed wave sources) or effectively infinite
pulses (for continuous wave sources).

Figure~\ref{fig:laser_fluc} illustrates the problem of light source
fluctuation. In this figure, each ``bundle'' represents approximately the
noise from the \ac{CCD}. If one considers this to bundle to represent the
width of the noise distribution due to the \ac{CCD}, then two different
phenomena are seen to occur.

Over time, the intensity error fluctuates in both a high frequency range
(mostly the \ac{CCD} noise) and a lower frequency. This lower frequency is
a result of the drift in the intensity of the output of the supercontinuum
laser. As such, it is simple to see that even a measurement within five
minutes of the blank sample is prone to intensity fluctuation error.

The second phenomena is that while most of the high frequency noise is due to
the \ac{CCD}, the stretching of the bundles suggests that there is a high
frequency noise component of the supercontinuum source as well. This means
that the fluctuation of the light source skews the distribution due to the
\ac{CCD} noise, sometimes broadening the width (such as in the second, third
and eighth bundles) and sometimes compressing the noise distribution (as seen
in the second to last bundle).

Combined, these two effects of the intensity fluctuation of the laser lead
to errors in calculating $\alpha$ (due to drift between measurements) and
$\sigma_{\alpha}$ (due to fluctuations in $\sigma_{I}$ and $\sigma_{I_0}$). An
additional frustration arises when one considers that these fluctuations in
intensity are often wavelength dependent, so it is not simple to extrapolate
the error calculations in intensity from one wavelength to another.

If the blank and sample spectra are taken within a few minutes of each other,
these error effects are minimised, but still lead to unaccountable error in
the final absorption spectrum. Given that the calculation of the standard
deviations will take part of the high frequency noise into account, and
the fast acquisition will minimise the low frequency source of noise, for
most measurements one can ignore these sources of error. However, for a
better sensitivity, it is required that information about the light source
fluctuation over time is known through a reference intensity spectrum. Without
this information, it is impossible to give a true estimate of the sensitivity
of a \ac{BBCEAS} instrument.

\subsection{Intensity fluctuations as a result of turbulence}

\begin{figure}
\begin{center}
\includegraphics[width=\textwidth]{figures/water_relax.pdf}
\end{center}
\caption{Relaxation time inside the cuvette}
\label{fig:relax}
\end{figure}

An additional source of error that causes intensity fluctuations is, perhaps
surprisingly, the turbulence of the solution. This can be clearly seen in
Figure~\ref{fig:relax}, where the intensity detected fluctuates wildly
during the injection of liquid into the cuvette (in this case, water was
injected into a cavity containing water). One will notice that right after the
injection phase, the intensity value reaches a maximum, falls and then slowly
builds back up to the beginning level.

The fluctuation and slow build up is most likely due to the alignment of the
light within the cavity. Any optical cavity is very sensitive to deflections
of any sort. The turbulence in the cuvette seems to deflect the path of the
light slightly. This disrupts the measured intensity by changing the total
build up intensity of light within the cavity and by changing the coupling
efficiency of the resulting light with the fibre based grating spectrometer.

Luckily, once the injection of liquid has ended, the resulting intensity build
up is modelled extremely well by an exponential rise back to the default
value. Using this type of model, it is simple to calculate that for the
\ac{BBCEAS} design used in this report, the time required to wait after an
injection before the intensity returns to normal is approximately one minute.
Knowing this time constant, and the slow fluctuation of the light source over
time, we can predict an approximate window in which the highest quality,
lowest noise spectra are likely to be taken.
\marginpar{\vspace{-34em}\\ If you are wondering, the slight dip near the end of
Figure~\ref{fig:relax}is me hitting the optical table.}

\section{Mirror Reflectivity}\label{sec:mirror_considerations}

\begin{figure}
\begin{center}
\includegraphics[width=\textwidth]{figures/mirrors.jpg}
\end{center}
\caption{}
\label{fig:mirror}
\end{figure}

No matter what equation an experimenter decides to use to calculate the
absorption spectrum of a sample, the mirror reflectivity must be explicitly
known, as it appears in the form of $\tfrac{1-R}{d}$ most commonly in the
literature. This factor represents a correction to the path length to account
for the number of passes that occur within a cavity.

If one neglects to determine the mirror reflectivity of the cavity, as a
function of wavelength, then two anomalous actions occur. The simplest to
understand is that the calculated values of absorption can have an extra error
that can easily exceed error due to other sources. While a calibration against
a known concentration allows one to neglect this source of error when
attempting to determine the unknown concentration of an absorber, the values
calculated provide an inaccurate understanding of the actual absorption of a
particular vibronic transition.

The second, more peculiar result of not neglecting the mirror reflectivity
when calculating an absorption spectrum changes the \emph{shape} of the
spectrum, and can lead to artifacts in the spectrum as well as an alteration
of the shape of a transition. Attempting to fit an accurate voigt profile to a
transition in a spectrum would be an exercise in faith at best.

There are three ways to determine the mirror reflectivity curves. The first is
to simply have the manufacturer provide the information, or to digitalise a
graph provided through an \ac{FTS} type measurement. The second common method
is to use the \ac{BBCEAS} setup for a known concentration \ac{CRDS}
experiment, which, due to the self calibrating nature of \ac{CRDS}, allows one
to extract the mirror reflectivity curves.

A new technique to determine the mirror reflectivity curve is to take a
measurement of the light transmitted through no cavity versus the light
through only one mirror. One can compare the two spectra by division to
determine the shape of the mirror reflectivity curve.

\marginpar{Show example of this using \ac{LED} spectrum}
One can then take several methods to scale this shape to the correct sizes. If
one knows the average mirror reflectivity and the error from the manufacturer,
then it is possible to scale the mirror reflectivity shape to nearly the
correct values (although only theoretically). Alternatively, since the
intensity of light passing through the mirror necessitates a different
exposure time for the \ac{CCD}, it is possible to create a plot of the
intensity as a function of exposure time, and use a fit to this plot to guess
what the intensity values of the spectrum without the mirror would be if the
exposure was the same as for the non-mirrored spectrum.

\section{Summary}
Put chapter summary here.
