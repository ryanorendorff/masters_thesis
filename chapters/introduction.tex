\chapter*{Introduction} \addcontentsline{toc}{chapter}{Introduction} \label{chap:intro}

Within the past few decades, lanthanide elements have become an intensely
active area of research. These elements have desirable properties that can be
exploited for applications such as laser gain mediums, fluorescent indicators,
television phosphors, as well as many other applications \cite{Bunzli:2005ic}.
As such, a theoretical understanding of lanthanide elements is vital to
engineering practical uses for these elements.

Understanding the chemical and physical properties of these elements requires
experimental and theoretical investigations of the electron configuration and
interaction inside the electron cloud surrounding a lanthanide atom. A common
way to determine these electron configuration properties is through
spectroscopy, which allows experimenters to investigate the internal structure
of lanthanide atoms through the atom's absorption and emission of light.
Unfortunately, not all lanthanide ions are readily amendable to standard
spectroscopic techniques due to low probabilities of absorbing passing
photons.  Such is the case for europium.

The recent advancements of absorption spectroscopic methods has allowed finicky
elements like europium to be probed. In the past few years, several research
groups have shown that it is possible to probe low concentrations of soluble
absorbers in a liquid medium using cavity techniques to enhance the absorption
signal, allowing investigations of aqueous lanthanide solutions to be better
understood.

This report provides an introduction to absorption spectroscopy and some
preliminary experimental results on the investigations of aqueous europium ions
and coordination complexes. The information collected provides a pathway for
moving forward with investigations of europium complexes, and provides
potential use cases that combine absorption spectroscopy with europium
complexes for spectroscopic assays.

\section*{Objectives}

The objective of this report is to perform two functions. The first is to
validate previous \ac{BBCEAS} results, and perform an analysis of the errors
that occur when using a \ac{BBCEAS} instrument. The second purpose of this
report is to analyse the electronic configuration of europium from both a
theoretical and experimental standpoint in attempt to determine uses for
aqueous europium ions and complexes.

\section*{Outline}

The chapters in this report are laid out as follows.

\begin{description}
  \item[Chapter 1] provides an introduction to absorption spectroscopy, some
    common experimental techniques, and the advantages and disadvantages of
    these techniques relative to each other.
  \item[Chapter 2] provides an introduction to broadband cavity enhanced
    absorption spectroscopy, which was used for the investigations of aqueous
    europium ions and complexes. In addition, this chapter details the
    experimental setup, algorithms used for processing data, and a discussion
    of future improvements.
  \item[Chapter 3] discusses the theoretical understandings of europium
    electronic transitions and how these theories are reflected in the observed
    europium spectra.
  \item[Chapter 4] discusses calibration measurements that were performed, and
    several sources of error in the measurements collected.
  \item[Chapter 5] discusses experimental results from measuring
    aqueous europium ions.
  \item[Chapter 6] concludes the report with a discussion of the acquired
    results and steps to take in the future to both complete theoretical
    investigations of europium complexes and how to use the results of
    this report to build sensitive protein assays.
\end{description}
